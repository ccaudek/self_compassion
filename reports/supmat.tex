\documentclass[aps,floatfix,prl]{revtex4}
\usepackage{graphicx,subfigure,amsmath,bbm}

\begin{document}
\title{Supplementary Material for\\ ``Gapless excitations in strongly fluctuating superconducting wires''}

\author{{ Dganit Meidan$^1$, Bernd Rosenow$^2$, Yuval Oreg$^3$ and Gil Refael$^4$}\\
{\small \em $^1 $Dahlem Center for Complex Quantum Systems and Institut
f\"{u}r Theoretische Physik, Freie Universit\"{a}t Berlin, 14195
Berlin, Germany\\
$^2$Institut f\"ur Theoretische Physik, Universit\"at Leipzig, D-04103, Leipzig, Germany\\
$^3$Department of Condensed Matter Physics, Weizmann Institute of Science, Rehovot, 76100, ISRAEL\\
$^4$Department of Physics, California Institute of Technology,
Pasadena, California 91125, USA}}

\date{\today}
\maketitle



%\begin{appendix}
\section{Sample characteristics}
\label{sample-characteristics}

The SCS total score was negatively associated with neuroticism, \(b\) =
-0.59, \(SE\) = 0.03, Bayesian 95\% CI {[}-0.65, -0.53{]} (Neff et al.,
2007), and positively associated with extroversion, \(b\) = 0.10, \(SE\)
= 0.03, Bayesian 95\% CI {[}0.03, 0.17{]}, and with agreeableness, \(b\)
= 0.07, \(SE\) = 0.03, Bayesian 95\% CI {[}0.01, 0.12{]}. The SCS total
score grew with age, \(b\) = 0.32, \(SE\) = 0.04, Bayesian 95\% CI
{[}0.23, 0.40{]}. We found no evidence of an association with level of
education, years of experience as rescue worker, time gap since the last
training, and rate of rescue worker activity. Women manifested lower
levels of the SCS than men (Yarnell et al., 2019), \(\Delta \text{SCS}\)
= 0.28, Bayesian 95\% CI {[}0.13, 0.42{]}, Cohen's \emph{d} = 0.28.

PTG was positively associated with neuroticism, \(b\) = 0.19, \(SE\) =
0.05, Bayesian 95\% CI {[}0.10, 0.28{]}, extroversion, \(b\) = 0.24,
\(SE\) = 0.05, Bayesian 95\% CI {[}0.14, 0.33{]}, conscientiousness,
\(b\) = 0.10, \(SE\) = 0.04, Bayesian 95\% CI {[}0.01, 0.18{]}, age,
\(b\) = 0.15, \(SE\) = 0.04, Bayesian 95\% CI {[}0.06, 0.24{]} (see
Garnefski et al., 2008), and rate of rescue-worker activity, \(b\) =
0.08, \(SE\) = 0.04, Bayesian 95\% CI {[}0.00, 0.16{]}; moreover, males
showed lower levels of PTG than females, \(b\) = -0.20, \(SE\) = 0.08,
Bayesian 95\% CI {[}-0.36, -0.04{]}.

IES-R scores were positively associated with neuroticism, \(b\) = 0.18,
\(SE\) = 0.04, Bayesian 95\% CI {[}0.11, 0.25{]} (Inoue et al., 2006),
and educational level, \(b\) = -0.06, \(SE\) = 0.02, Bayesian 95\% CI
{[}-0.10, -0.01{]} (Wu et al., 2005), but negatively associated with
age, \(b\) = -0.05, \(SE\) = 0.03, Bayesian 95\% CI {[}-0.10, -0.00{]}.

\newpage

\section{SCS factor structure}\label{scs-factor-structure}

We compared the fit of all models discussed by Neff et al. (2019). We
started with the unidimensional CFA model, which clearly proved to be
inadequate, CFI = 0.61, TLI = 0.58, RMSEA = 0.18 {[}90\% CI
0.18-0.19{]}, SRMR = 0.16, \(\omega_t\) = 0.44. We then considered all
the factor structures that had been discussed by Neff et al. (2019). The
list of the examined models, together with their fit indexes, is provide
below:

\begin{itemize}
\item
  a two-factor CFA for the positive and the negative components of SC,
  CFI = 0.83, TLI = 0.82, RMSEA = 0.12 {[}90\% CI 0.12-0.12{]}, SRMR =
  0.10, \(\omega_t\) = 0.96;
\item
  a two-factor ESEM, CFI = 0.85, TLI = 0.82, RMSEA = 0.12 {[}90\% CI
  0.12-0.12{]}, SRMR = 0.06, \(\omega_t\) = 0.96;
\item
  a six-factor CFA, CFI = 0.90, TLI = 0.88, RMSEA = 0.10 {[}90\% CI
  0.09-0.10{]}, SRMR = 0.07, \(\omega_t\) = 0.97;
\item
  a six-factor ESEM, CFI = 0.98, TLI = 0.96, RMSEA = 0.05 {[}90\% CI
  0.05-0.06{]}, SRMR = 0.02, \(\omega_t\) = 0.98;
\item
  a bifactor-CFA (1 G- and 6 S-factors), CFI = 0.76, TLI = 0.71, RMSEA =
  0.15 {[}90\% CI 0.15-0.16{]}, SRMR = 0.12, \(\omega_t\) = 0.97;
\item
  a bifactor-ESEM (1 G- and 6 S-factors), CFI = 0.98, TLI = 0.96, RMSEA
  = 0.05 {[}90\% CI 0.05-0.06{]}, SRMR = 0.02, \(\omega_t\) = 0.99;
\item
  a two-bifactor (two-tier) CFA model (2 G- and 6 S-factors), CFI =
  0.90, TLI = 0.88, RMSEA = 0.10 {[}90\% CI 0.10-0.10{]}, SRMR = 0.08,
  \(\omega_t\) = 0.99;
\item
  a two-bifactor (two-tier) ESEM model (2 G- and 6 S-factors), CFI =
  0.99, TLI = 0.98, RMSEA = 0.04 {[}90\% CI 0.04-0.05{]}, SRMR = 0.01,
  \(\omega_t\) = 0.99.
\end{itemize}

For the two-factor CFA for the CS and RUS of SC, the correlation between
the factors was -0.28. For the two-factor ESEM, the correlation between
the factors was -0.21. For the 6-factor CFA, the correlations between
factors ranged between -0.44 and 0.95. For the 6-factor ESEM, the
correlations between factors ranged between -0.36 and 0.58. For the
two-bifactor (two-tier) CFA model (2 G- and 6 S-factors), the
correlation between the CS and RUS factors is -0.34. For the
two-bifactor (two-tier) ESEM model (2 G- and 6 S-factors), the
correlation between the CS and RUS factors is -0.55.

In our sample, the two-bifactor (two-tier) ESEM model, which included 6
uncorrelated factors (self-kindness, reduced self-judgment, common
humanity, reduced isolation, mindfulness, and reduced
over-identification) and two correlated CS (loading on the
self-kindness, common humanity, and mindfulness items) and RUS (loading
on the reduced self-judgment, reduced isolation, and reduced
over-identification items) factors showed a marginally better fit than
the one bifactor-ESEM (1 G- and 6 S-factors) model. What is crucial is
that, among the considered models, those that provide the best fits to
the data include distinct factors corresponding to the six dimensions of
the SCS (self-kindness, reduced self-judgment, common humanity, reduced
isolation, mindfulness, and reduced over-identification).

\newpage


\section{A two-factor correlated model with two unitary factors
representing CS and
RUS}\label{a-two-factor-correlated-model-with-two-unitary-factors-representing-cs-and-rus}

\subsection{Mplus syntax}\label{mplus-syntax}

\begin{verbatim}
TITLE: 2 factor ESEM.
DATA:
  FILE IS selfcompassionitems.dat;
VARIABLE:
  NAMES ARE scsj1 scoi2 scch3 scis4 scsk5 scoi6 scch7 scsj8 
            scmi9 scch10 scsj11 scsk12 scis13 scmi14 scch15 
            scsj16 scmi17 scis18 scsk19 scoi20 scsj21 scmi22 
            scsk23 scoi24 scis25 scsk26; 
  MISSING ARE ALL(-9);
  USEVARIABLES ARE scsj1 scoi2 scch3 scis4 scsk5 scoi6 scch7 scsj8 
            scmi9 scch10 scsj11 scsk12 scis13 scmi14 scch15 
            scsj16 scmi17 scis18 scsk19 scoi20 scsj21 scmi22 
            scsk23 scoi24 scis25 scsk26; 
  CATEGORICAL ARE all;
ANALYSIS:
  ! Requesting the weighted least squares mean- and variance-adjusted estimator
  ESTIMATOR = WLSMV;
  ROTATION = target;
MODEL:
  pos BY scsk5 scsk12 scsk19 scsk23 scsk26 scsj1~0 scsj8~0 scsj11~0 
         scsj16~0 scsj21~0 scch3 scch7 scch10 scch15 scis4~0 scis13~0 
         scis18~0 scis25~0 scmi9 scmi14 scmi17 scmi22 scoi2~0 scoi6~0 
         scoi20~0 scoi24~0 (*1);
  neg BY scsk5~0 scsk12~0 scsk19~0 scsk23~0 scsk26~0 scsj1 scsj8 scsj11 
         scsj16 scsj21 scch3~0 scch7~0 scch10~0 scch15~0 scis4 scis13 
         scis18 scis25 scmi9~0 scmi14~0 scmi17~0 scmi22~0 scoi2 scoi6 
         scoi20 scoi24 (*1);
! Requesting standardized parameter estimates
OUTPUT: stdyx;
\end{verbatim}

\newpage


\subsubsection{Standardized Factor
Loadings}\label{standardized-factor-loadings}

\captionsetup[table]{labelformat=empty,skip=1pt}
\begin{longtable}{crr}
\toprule
Items & CS & RUS \\ 
\midrule
\multicolumn{1}{l}{Self-kindness} \\ 
\midrule
SCSK05 & 0.69 & 0.12 \\ 
SCSK12 & 0.76 & -0.04 \\ 
SCSK19 & 0.76 & -0.08 \\ 
SCSK23 & 0.60 & -0.27 \\ 
SCSK26 & 0.66 & -0.11 \\ 
\midrule
\multicolumn{1}{l}{Mindfulness} \\ 
\midrule
SCMI09 & 0.52 & 0.06 \\ 
SCMI14 & 0.56 & -0.19 \\ 
SCMI17 & 0.63 & -0.23 \\ 
SCMI22 & 0.66 & -0.08 \\ 
\midrule
\multicolumn{1}{l}{Common Humanity} \\ 
\midrule
SCCH03 & 0.46 & 0.23 \\ 
SCCH07 & 0.57 & 0.40 \\ 
SCCH10 & 0.62 & 0.30 \\ 
SCCH15 & 0.69 & 0.00 \\ 
\midrule
\multicolumn{1}{l}{Self-judgment} \\ 
\midrule
SCSJ01 & 0.07 & 0.55 \\ 
SCSJ08 & 0.02 & 0.77 \\ 
SCSJ11 & 0.03 & 0.74 \\ 
SCSJ16 & -0.01 & 0.74 \\ 
SCSJ21 & 0.03 & 0.60 \\ 
\midrule
\multicolumn{1}{l}{Isolation} \\ 
\midrule
SCIS04 & -0.08 & 0.82 \\ 
SCIS13 & -0.04 & 0.78 \\ 
SCIS18 & 0.02 & 0.82 \\ 
SCIS25 & -0.05 & 0.78 \\ 
\midrule
\multicolumn{1}{l}{Over-identification} \\ 
\midrule
SCOI02 & -0.07 & 0.81 \\ 
SCOI06 & -0.04 & 0.81 \\ 
SCOI20 & 0.04 & 0.67 \\ 
SCOI24 & -0.03 & 0.68 \\ 
\bottomrule
\end{longtable}



\newpage

\hypertarget{a-two-factor-correlated-model-with-two-unitary-factors-representing-cs-and-rus-1}{%
\section{A two-factor correlated model with two unitary factors
representing CS and
RUS}\label{a-two-factor-correlated-model-with-two-unitary-factors-representing-cs-and-rus-1}}

\hypertarget{mplus-syntax-r-factor-extraction}{%
\subsection{Mplus syntax -- R factor
extraction}\label{mplus-syntax-r-factor-extraction}}

\begin{verbatim}
TITLE: 2 factor ESEM v2.
DATA:
  FILE IS selfcompassionitems.dat;
VARIABLE:
  NAMES ARE scsj1 scoi2 scch3 scis4 scsk5 scoi6 scch7 scsj8 
            scmi9 scch10 scsj11 scsk12 scis13 scmi14 scch15 
            scsj16 scmi17 scis18 scsk19 scoi20 scsj21 scmi22 
            scsk23 scoi24 scis25 scsk26; 
  MISSING ARE ALL(-9);
  USEVARIABLES ARE scsj1 scoi2 scch3 scis4 scsk5 scoi6 scch7 scsj8 
            scmi9 scch10 scsj11 scsk12 scis13 scmi14 scch15 
            scsj16 scmi17 scis18 scsk19 scoi20 scsj21 scmi22 
            scsk23 scoi24 scis25 scsk26; 
  CATEGORICAL ARE all;
ANALYSIS: 
  ESTIMATOR = WLSMV; 
  ROTATION = oblimin; 
MODEL:
  pos BY scsj1@.06 scoi2@-.05 scch3@.41 scis4@-.070 scsk5@.65 scoi6@-.03 
        scch7@0.49 scsj8@.01 scmi9@0.46 scch10@0.55 scsj11@.02 scsk12@.69 
        scis13@-.04 scmi14@0.47 scch15@0.66 scsj16@-0.01 scmi17@0.58 
        scis18@0.03 scsk19@0.70 scoi20@0.07 scsj21@0.02 scmi22@0.65 
        scsk23@0.56 scoi24@-0.01 scis25@-0.04 scsk26@0.64;
  neg BY scsj1@.49 scoi2@.78 scch3@.22 scis4@.78 scsk5@.13 scoi6@.79 
        scch7@.33 scsj8@.70 scmi9@.07 scch10@.24 scsj11@.68 scsk12@-.03 
        scis13@.71 scmi14@-.17 scch15@.02 scsj16@.69 scmi17@-.2 
        scis18@.77 scsk19@-.06 scoi20@.64 scsj21@.56 scmi22@-.05 
        scsk23@-.22 scoi24@0.62 scis25@.75 scsk26@-.06;
  pos@1;
  neg@1;
OUTPUT: stdyx;
\end{verbatim}

\newpage

\hypertarget{standardized-factor-loadings-1}{%
\subsubsection{Standardized Factor
Loadings}\label{standardized-factor-loadings-1}}

\captionsetup[table]{labelformat=empty,skip=1pt}
\begin{longtable}{crr}
\toprule
Items & CS & RUS \\ 
\midrule
\multicolumn{1}{l}{Self-kindness} \\ 
\midrule
SCSK05 & 0.65 & 0.13 \\ 
SCSK12 & 0.69 & -0.03 \\ 
SCSK19 & 0.70 & -0.06 \\ 
SCSK23 & 0.56 & -0.22 \\ 
SCSK26 & 0.64 & -0.06 \\ 
\midrule
\multicolumn{1}{l}{Mindfulness} \\ 
\midrule
SCMI09 & 0.46 & 0.07 \\ 
SCMI14 & 0.47 & -0.17 \\ 
SCMI17 & 0.58 & -0.20 \\ 
SCMI22 & 0.65 & -0.05 \\ 
\midrule
\multicolumn{1}{l}{Common Humanity} \\ 
\midrule
SCCH03 & 0.41 & 0.22 \\ 
SCCH07 & 0.49 & 0.33 \\ 
SCCH10 & 0.55 & 0.24 \\ 
SCCH15 & 0.66 & 0.02 \\ 
\midrule
\multicolumn{1}{l}{Self-judgment} \\ 
\midrule
SCSJ01 & 0.06 & 0.49 \\ 
SCSJ08 & 0.01 & 0.70 \\ 
SCSJ11 & 0.02 & 0.68 \\ 
SCSJ16 & -0.01 & 0.69 \\ 
SCSJ21 & 0.02 & 0.56 \\ 
\midrule
\multicolumn{1}{l}{Isolation} \\ 
\midrule
SCIS04 & -0.07 & 0.78 \\ 
SCIS13 & -0.04 & 0.71 \\ 
SCIS18 & 0.03 & 0.77 \\ 
SCIS25 & -0.04 & 0.75 \\ 
\midrule
\multicolumn{1}{l}{Over-identification} \\ 
\midrule
SCOI02 & -0.05 & 0.78 \\ 
SCOI06 & -0.03 & 0.79 \\ 
SCOI20 & 0.07 & 0.64 \\ 
SCOI24 & -0.01 & 0.62 \\ 
\bottomrule
\end{longtable}



\newpage

\hypertarget{a-two-correlated-factors-bifactor-model}{%
\section{A two-correlated factors bifactor
model}\label{a-two-correlated-factors-bifactor-model}}

\hypertarget{mplus-syntax-1}{%
\subsection{Mplus syntax}\label{mplus-syntax-1}}

\begin{verbatim}
TITLE: bifactor 2-correlated-factors ESEM.
DATA:
  FILE IS selfcompassionitems.dat;
VARIABLE:
  NAMES ARE scsj1 scoi2 scch3 scis4 scsk5 scoi6 scch7 scsj8
            scmi9 scch10 scsj11 scsk12 scis13 scmi14 scch15
            scsj16 scmi17 scis18 scsk19 scoi20 scsj21 scmi22
            scsk23 scoi24 scis25 scsk26;
  MISSING ARE ALL(-9);
  USEVARIABLES ARE scsj1 scoi2 scch3 scis4 scsk5 scoi6 scch7 scsj8
            scmi9 scch10 scsj11 scsk12 scis13 scmi14 scch15
            scsj16 scmi17 scis18 scsk19 scoi20 scsj21 scmi22
            scsk23 scoi24 scis25 scsk26;
  CATEGORICAL ARE all;
ANALYSIS:
  ! Requesting the weighted least squares mean- and variance-adjusted estimator
  ESTIMATOR = WLSMV;
  ROTATION = target;
MODEL:
  sc BY scsk5 scsk12 scsk19 scsk23 scsk26 scsj1 scsj8 scsj11
        scsj16 scsj21 scch3 scch7 scch10 scch15 scis4 scis13
        scis18 scis25 scmi9 scmi14 scmi17 scmi22 scoi2 scoi6
        scoi20 scoi24(*1);
  pos BY scsk5 scsk12 scsk19 scsk23 scsk26 scsj1~0 scsj8~0 scsj11~0
         scsj16~0 scsj21~0 scch3 scch7 scch10 scch15 scis4~0 scis13~0
         scis18~0 scis25~0 scmi9 scmi14 scmi17 scmi22 scoi2~0 scoi6~0
         scoi20~0 scoi24~0 (*1);
  neg BY scsk5~0 scsk12~0 scsk19~0 scsk23~0 scsk26~0 scsj1 scsj8 scsj11
         scsj16 scsj21 scch3~0 scch7~0 scch10~0 scch15~0 scis4 scis13
         scis18 scis25 scmi9~0 scmi14~0 scmi17~0 scmi22~0 scoi2 scoi6
         scoi20 scoi24 (*1);
! Requesting standardized parameter estimates
OUTPUT: stdyx;
\end{verbatim}

\newpage

\hypertarget{standardized-factor-loadings-2}{%
\subsubsection{Standardized Factor
Loadings}\label{standardized-factor-loadings-2}}

\captionsetup[table]{labelformat=empty,skip=1pt}
\begin{longtable}{crrr}
\toprule
Items & GEN & CS & RUS \\ 
\midrule
\multicolumn{1}{l}{Self-kindness} \\ 
\midrule
SCSK05 & 0.19 & 0.70 & 0.08 \\ 
SCSK12 & 0.15 & 0.76 & -0.09 \\ 
SCSK19 & 0.13 & 0.77 & -0.12 \\ 
SCSK23 & 0.09 & 0.62 & -0.30 \\ 
SCSK26 & 0.12 & 0.68 & -0.14 \\ 
\midrule
\multicolumn{1}{l}{Mindfulness} \\ 
\midrule
SCMI09 & -0.46 & 0.46 & 0.13 \\ 
SCMI14 & -0.60 & 0.47 & -0.11 \\ 
SCMI17 & -0.46 & 0.59 & -0.18 \\ 
SCMI22 & -0.17 & 0.65 & -0.06 \\ 
\midrule
\multicolumn{1}{l}{Common Humanity} \\ 
\midrule
SCCH03 & -0.22 & 0.43 & 0.26 \\ 
SCCH07 & 0.09 & 0.58 & 0.38 \\ 
SCCH10 & -0.04 & 0.62 & 0.29 \\ 
SCCH15 & -0.21 & 0.67 & 0.02 \\ 
\midrule
\multicolumn{1}{l}{Self-judgment} \\ 
\midrule
SCSJ01 & -0.25 & 0.03 & 0.59 \\ 
SCSJ08 & -0.19 & -0.01 & 0.80 \\ 
SCSJ11 & -0.14 & 0.00 & 0.76 \\ 
SCSJ16 & -0.10 & -0.04 & 0.76 \\ 
SCSJ21 & -0.14 & 0.01 & 0.64 \\ 
\midrule
\multicolumn{1}{l}{Isolation} \\ 
\midrule
SCIS04 & 0.24 & -0.07 & 0.79 \\ 
SCIS13 & 0.40 & -0.01 & 0.72 \\ 
SCIS18 & 0.36 & 0.05 & 0.77 \\ 
SCIS25 & 0.24 & -0.04 & 0.76 \\ 
\midrule
\multicolumn{1}{l}{Over-identification} \\ 
\midrule
SCOI02 & 0.22 & -0.06 & 0.79 \\ 
SCOI06 & 0.19 & -0.04 & 0.80 \\ 
SCOI20 & 0.37 & 0.08 & 0.62 \\ 
SCOI24 & 0.36 & 0.00 & 0.63 \\ 
\bottomrule
\end{longtable}



\newpage

\hypertarget{correlated-residuals-between-items-scch10-and-scch7-and-between-items-scis18-and-scis13}{%
\subsection{Correlated residuals between items SCCH10 and SCCH7, and
between items SCIS18 and
SCIS13}\label{correlated-residuals-between-items-scch10-and-scch7-and-between-items-scis18-and-scis13}}

\hypertarget{mplus-syntax-2}{%
\subsubsection{Mplus syntax}\label{mplus-syntax-2}}

\begin{verbatim}
TITLE: 2 factor bi-factor ESEM with correlated residuals.
DATA:
  FILE IS selfcompassionitems.dat;
VARIABLE:
  NAMES ARE scsj1 scoi2 scch3 scis4 scsk5 scoi6 scch7 scsj8 
            scmi9 scch10 scsj11 scsk12 scis13 scmi14 scch15 
            scsj16 scmi17 scis18 scsk19 scoi20 scsj21 scmi22 
            scsk23 scoi24 scis25 scsk26; 
  MISSING ARE ALL(-9);
  USEVARIABLES ARE scsj1 scoi2 scch3 scis4 scsk5 scoi6 scch7 scsj8 
            scmi9 scch10 scsj11 scsk12 scis13 scmi14 scch15 
            scsj16 scmi17 scis18 scsk19 scoi20 scsj21 scmi22 
            scsk23 scoi24 scis25 scsk26; 
  CATEGORICAL ARE all;
ANALYSIS:
  ! Requesting the weighted least squares mean- and variance-adjusted estimator
  ESTIMATOR = WLSMV;
  ROTATION = target;
MODEL:
  sc BY scsk5 scsk12 scsk19 scsk23 scsk26 scsj1 scsj8 scsj11 
        scsj16 scsj21 scch3 scch7 scch10 scch15 scis4 scis13 
        scis18 scis25 scmi9 scmi14 scmi17 scmi22 scoi2 scoi6 
        scoi20 scoi24 (*1);
  pos BY scsk5 scsk12 scsk19 scsk23 scsk26 scsj1~0 scsj8~0 scsj11~0 
         scsj16~0 scsj21~0 scch3 scch7 scch10 scch15 scis4~0 scis13~0 
         scis18~0 scis25~0 scmi9 scmi14 scmi17 scmi22 scoi2~0 scoi6~0 
         scoi20~0 scoi24~0 (*1);
  neg BY scsk5~0 scsk12~0 scsk19~0 scsk23~0 scsk26~0 scsj1 scsj8 scsj11 
         scsj16 scsj21 scch3~0 scch7~0 scch10~0 scch15~0 scis4 scis13 
         scis18 scis25 scmi9~0 scmi14~0 scmi17~0 scmi22~0 scoi2 scoi6 
         scoi20 scoi24 (*1);
  scch10 WITH scch7;
  scis18 WITH scis13;
! Requesting standardized parameter estimates
OUTPUT: stdyx;
\end{verbatim}

\newpage

\hypertarget{standardized-factor-loadings-3}{%
\subsubsection{Standardized Factor
Loadings}\label{standardized-factor-loadings-3}}

\captionsetup[table]{labelformat=empty,skip=1pt}
\begin{longtable}{crrr}
\toprule
Items & GEN & CS & RUS \\ 
\midrule
\multicolumn{1}{l}{Self-kindness} \\ 
\midrule
SCSK05 & -0.20 & 0.71 & 0.10 \\ 
SCSK12 & -0.15 & 0.78 & -0.06 \\ 
SCSK19 & -0.13 & 0.78 & -0.10 \\ 
SCSK23 & -0.09 & 0.63 & -0.28 \\ 
SCSK26 & -0.12 & 0.69 & -0.12 \\ 
\midrule
\multicolumn{1}{l}{Mindfulness} \\ 
\midrule
SCMI09 & 0.47 & 0.46 & 0.13 \\ 
SCMI14 & 0.60 & 0.47 & -0.12 \\ 
SCMI17 & 0.46 & 0.59 & -0.18 \\ 
SCMI22 & 0.16 & 0.66 & -0.05 \\ 
\midrule
\multicolumn{1}{l}{Common Humanity} \\ 
\midrule
SCCH03 & 0.23 & 0.43 & 0.26 \\ 
SCCH07 & -0.09 & 0.50 & 0.34 \\ 
SCCH10 & 0.08 & 0.54 & 0.25 \\ 
SCCH15 & 0.22 & 0.68 & 0.03 \\ 
\midrule
\multicolumn{1}{l}{Self-judgment} \\ 
\midrule
SCSJ01 & 0.25 & 0.03 & 0.59 \\ 
SCSJ08 & 0.20 & -0.02 & 0.80 \\ 
SCSJ11 & 0.15 & 0.00 & 0.76 \\ 
SCSJ16 & 0.11 & -0.04 & 0.76 \\ 
SCSJ21 & 0.15 & 0.01 & 0.64 \\ 
\midrule
\multicolumn{1}{l}{Isolation} \\ 
\midrule
SCIS04 & -0.24 & -0.07 & 0.80 \\ 
SCIS13 & -0.28 & -0.04 & 0.68 \\ 
SCIS18 & -0.24 & 0.03 & 0.74 \\ 
SCIS25 & -0.23 & -0.03 & 0.77 \\ 
\midrule
\multicolumn{1}{l}{Over-identification} \\ 
\midrule
SCOI02 & -0.22 & -0.05 & 0.80 \\ 
SCOI06 & -0.19 & -0.03 & 0.80 \\ 
SCOI20 & -0.38 & 0.10 & 0.63 \\ 
SCOI24 & -0.37 & 0.02 & 0.64 \\ 
\bottomrule
\end{longtable}



\newpage

\hypertarget{latent-profile-analysis}{%
\section{Latent Profile Analysis}\label{latent-profile-analysis}}

Prior to the analysis, the six self-compassion dimensions were
standardized and scores on the three RUS SCS sub-scales were reversed
(\emph{i.e.}, they were indicators of ``lack of'' Self judgment,
Overidentification, and Isolation). By following Ullrich-French \& Cox
(2020), we select the best LPA model of the SCS by specifying 1 through
6 profiles. Model selection was based on an analytic hierarchy process
resting on the comparison of fit indexes. The best solution was a model
with 6 classes. All models were fit in MPLUS 8.6 and freely estimated
the means and variances of indicators with robust maximum likelihood.

Two multilevel Bayesian regressions models were run with either PTG or
IES-R scores as the dependent variable and group membership according to
the six profile solution as independent variable.

All contrasts between PTG or IES-R mean pairs were evaluated with the
Tukey correction. The resulting HPD 95\% intervals not including the
zero point were coded with 1, if they were consistent with the
prediction formulated according to the relevant dimensions specified by
H1 or H2, and with 0 if they were not. For example, let us consider the
contrast between the profiles 3 and 6 described in
Fig.~@ref(fig:figureLPA) for the PTG dependent variable. According to
H1, the mean difference in PTG scores should only depend on the SC
components of the SCS scale. Therefore, profile 6 (\emph{High CS Medium
RUS}) is expected to have a higher PTG mean value than profile 3
(\emph{Low CS High RUS}). This prediction was satisfied in the present
sample (0.21 vs.~-0.32) and, therefore, we coded the results of this
contrast as 1 (``success''). If we compare these same two profiles in
terms of H2 (\emph{i.e.}, the mean difference in PTG scores should only
depend on the RUS components of the SCS scale), then we should expect a
higher PTG score in profile 3 than in profile 6. This did not happen
and, therefore, in terms of H2, this contrast was coded as 0.

Adjusted posterior means were then computed with the \texttt{emmeans} R
package. The procedure described in the example provided in the
manuscript was applied to each of the 13 contrasts with HPD 95\%
intervals not including the zero point (by considering both for the
contrasts computed with PTG as the dependent variable and the contrasts
computed with IES-R as the dependent variable). In this manner, we
obtained 13 out of 13 successes when ``success'' was coded according to
H1, and 5 out of 13 successes when ``success'' was coded according to
H2.

The analysis of this difference in terms of ``proportion correct''
produced a posterior median difference of 4.16 with a MAD standard
deviation of 1.49 (or, in frequentist terms, \(\Delta\) prop. = 0.62,
\emph{SE} = 0.13, \emph{p} = 0.00), which indicates a reliable
difference. We interpret this result as supporting H1 (\emph{i.e.}, the
CS and RUS components have functionally different purposes) rather than
H2 (\emph{i.e.}, the six sub-scales of the SCS do not measure
functionally different dimensions of the construct).

\newpage

\hypertarget{sem-models}{%
\section{SEM models}\label{sem-models}}

\hypertarget{model-m0}{%
\subsection{Model M0}\label{model-m0}}

M0 considers two endogenous variables: post-traumatic growth (ptgr) and
post-traumatic stress (ptss) and their relations with 4 exogenous
variables: coping (cope), perceived social support (soc),
self-compassion (sc), and neuroticism (neuro). In model M0, only the
regression effects of cope and soc are considered. Other two variables
also included (self-compassion and neuroticism), althought they have no
effect on the endogenous variables, in order to allow comparisons
between nested models.

Model 0 comprised six latent factors: self-compassion, coping, perceived
social support, Neuroticism, post-traumatic growth, and post-traumatic
stress disorder. Each latent variable was identified by its sub-scale
scores as indicators. Self-compassion was represented as a unitary
construct with six indicators. Model 0 included direct paths between two
exogenous variables (coping, perceived social support) and the two
endogenous variables of interest (post-traumatic growth, and
post-traumatic stress disorder). No direct paths were specified between
both self-compassion and neuroticism and the two exogenous variables.

\hypertarget{models-definition-lavaan-syntax}{%
\subsubsection{Model's definition (lavaan
syntax)}\label{models-definition-lavaan-syntax}}

\begin{verbatim}
model0 <- "
  # post-traumatic growth
  ptgr =~ life_appreciation + new_possibilities + 
          personal_strength + spirituality_changes + 
          interpersonal_relationships
  # pts
  ptss =~ avoiding + intrusivity + iperarousal
  # coping
  cope =~ social_support + avoiding_strategies + 
          positive_attitude + problem_orientation + 
          transcendent_orientation
  # perceived social support
  soc =~ family + friends + significant_other
  # self-compassion
  sc =~ self_judgment + isolation + over_identification +
        self_kindness + common_humanity + mindfulness
  # neuroticism
  neuro =~ negative_affect + self_reproach
  # regressions
  ptgr ~ cope + soc  
  ptss ~ cope + soc 
  # Residual correlations
  self_judgment ~~ self_kindness
  "
\end{verbatim}

Modification indexes suggested the inclusion of a residual covariance
between the subscales of Self judgment and Self kindness. Also with this
specification, Model 0 showed an unacceptable fit with the data,
\(\chi^2\)(240) = 2,484.79, \(\chi^2\)/df = 10.35, CFI = 0.76, NFI =
0.74, TLI = 0.72, RMSEA = 0.11, and SRMS = 0.15. Overall, the CFA fit
indexes did not support Model 0, which did not include regression
effects for both self-compassion and neuroticism.

\newpage

\hypertarget{model-1}{%
\subsection{Model 1}\label{model-1}}

M1 considers, besides the regression effects of M0, also an effect of
self-compassion, but without distinguishing the RUS and CS components.
Modification indexes suggested the addition of a residual correlation
between Self-judgment and Self-kindness.

\hypertarget{models-definition-lavaan-syntax-1}{%
\subsubsection{Model's definition (lavaan
syntax)}\label{models-definition-lavaan-syntax-1}}

In model M1, self-compassion was conceived as a unitary construct
defined by six indicators (\emph{i.e.}, no distinction between CS and
RUS was made). Model 1 comprised two additional direct regression paths
between self compassion and the two endogenous variables, which improved
model fit compared to Model 0, \(\Delta \chi^2\)(2) = 123, \(p =\) 0.
However, the overall model fit was still unacceptable, \(\chi^2\)(238) =
2,302.08, \(\chi^2\)/df = 9.67, CFI = 0.78, NFI = 0.76, TLI = 0.74,
RMSEA = 0.11, and SRMS = 0.13.

\begin{verbatim}
model1 <- "
  # post-traumatic growth
  ptgr =~ life_appreciation + new_possibilities + 
          personal_strength + spirituality_changes + 
          interpersonal_relationships
  # ptsd
  ptss =~ avoiding + intrusivity + iperarousal
  # coping
  cope =~ social_support + avoiding_strategies + 
          positive_attitude + problem_orientation + 
          transcendent_orientation
  # perceived social support
  soc =~ family + friends + significant_other
  # self-compassion
  sc =~ self_judgment + isolation + over_identification +
        self_kindness + common_humanity + mindfulness
  # neuroticism
  neuro =~ negative_affect + self_reproach
  # regressions
  ptgr ~ cope + soc + sc 
  ptss ~ cope + soc + sc 
  # residual correlations
  self_judgment ~~ self_kindness
  "
\end{verbatim}

\newpage

\hypertarget{model-1a}{%
\subsection{Model 1a}\label{model-1a}}

Model 1a attempted to improve the fit of Model 1 by including only a
subset of indicators for Coping (\emph{i.e.}, Positive attitude and
Problem orientation), because Coping was poorly defined by the other
indicators. This modification improved the model fit substantially,
\(\Delta \chi^2\)(63) = 850.08, \(p =\) 0. However, the overall model
fit was still unacceptable, \(\chi^2\)(175) = 1,389.15, \(\chi^2\)/df =
7.94, CFI = 0.85, NFI = 0.84, TLI = 0.83, RMSEA = 0.10, and SRMS = 0.10.

\hypertarget{models-definition-lavaan-syntax-2}{%
\subsubsection{Model's definition (lavaan
syntax)}\label{models-definition-lavaan-syntax-2}}

\begin{verbatim}
model1a <- "
  # post-traumatic growth
  ptgr =~ life_appreciation + new_possibilities + 
         personal_strength + spirituality_changes + 
         interpersonal_relationships
  # ptsd
  ptss =~ avoiding + intrusivity + iperarousal
  # coping
  cope =~ positive_attitude + problem_orientation 
  # perceived social support
  soc =~ family + friends + significant_other
  # self-compassion
  sc =~ self_judgment + isolation + over_identification +
        self_kindness + common_humanity + mindfulness
  # neuroticism
  neuro =~ negative_affect + self_reproach
  sc ~~ neuro
  soc ~~ cope
  soc ~~ sc
  soc ~~ neuro
  cope ~~ sc
  cope ~~ neuro
  # regressions
  ptgr ~ cope + soc + sc 
  ptss ~ cope + soc + sc 
  # residual correlations
  self_judgment ~~ self_kindness
"
\end{verbatim}

\newpage

\hypertarget{model-2}{%
\subsection{Model 2}\label{model-2}}

Model 2 was identical to Model 1a, except from the fact that, instead of
having a unitary self-compassion latent variable, the ``lack of'' Self
judgment, Overidentification, and Isolation indicators revealed the RUS
latent variable, and the Self-kindness, Common humanity, and Mindfulness
indicators revealed the CS latent variable. Model 2 represented an
improvement of fit relative to Model 1a, \(\Delta \chi^2\)(6) = 626.20,
\(p =\) 0. Overall, the model fit indexes underlined a good fit of Model
2, \(\chi^2\)(169) = 618.93, \(\chi^2\)/df = 3.66, CFI = 0.95, NFI =
0.93, TLI = 0.93, RMSEA = 0.06, and SRMS = 0.06.

\hypertarget{models-definition-lavaan-syntax-3}{%
\subsubsection{Model's definition (lavaan
syntax)}\label{models-definition-lavaan-syntax-3}}

\begin{verbatim}
model2 <- "
  # post-traumatic growth
  ptgr =~ life_appreciation + new_possibilities + 
         personal_strength + spirituality_changes + 
         interpersonal_relationships
  # ptsd
  ptss =~ avoiding + intrusivity + iperarousal
  # coping
  cope =~ positive_attitude + problem_orientation 
  # perceived social support
  soc =~ family + friends + significant_other
  # self-compassion
  nsc =~ self_judgment + isolation + over_identification
  psc =~ self_kindness + common_humanity + mindfulness
  # neuroticism
  neuro =~ negative_affect + self_reproach
  psc ~~ nsc
  psc ~~ neuro
  nsc ~~ neuro
  soc ~~ cope
  soc ~~ nsc
  soc ~~ psc
  soc ~~ neuro
  cope ~~ nsc
  cope ~~ psc
  cope ~~ neuro
  # regressions
  ptgr ~ cope + soc + nsc + psc 
  ptss ~ cope + soc + nsc + psc 
  # residual correlations
  self_judgment ~~ self_kindness
"
\end{verbatim}

\newpage

\hypertarget{model-3}{%
\section{Model 3}\label{model-3}}

Model M3 adds the regression coefficient for Neuroticism.

\begin{verbatim}
model3 <- "
  # post-traumatic growth
  ptgr =~ life_appreciation + new_possibilities + 
         personal_strength + spirituality_changes + 
         interpersonal_relationships
  # ptsd
  ptss =~ avoiding + intrusivity + iperarousal
  # coping
  cope =~ positive_attitude + problem_orientation 
  # perceived social support
  soc =~ family + friends + significant_other
  # self-compassion
  psc =~ self_kindness + common_humanity + mindfulness
  nsc =~ self_judgment + isolation + over_identification
  # neuroticism
  neuro =~ negative_affect + self_reproach
  # regressions
  ptss ~ cope + soc + nsc + psc + neuro
  ptgr ~ cope + soc + nsc + psc + neuro
  # residual correlations
  self_judgment ~~ self_kindness
"
\end{verbatim}

\newpage

\hypertarget{model-4}{%
\section{Model 4}\label{model-4}}

Model 4 remove the two self-compassion regression effects from M3.

\begin{verbatim}
model4 <- "
  # post-traumatic growth
  ptgr =~ life_appreciation + new_possibilities + 
         personal_strength + spirituality_changes + 
         interpersonal_relationships
  # ptss
  ptss =~ avoiding + intrusivity + iperarousal
  # coping
  cope =~ positive_attitude + problem_orientation 
  # perceived social support
  soc =~ family + friends + significant_other
  # neuroticism
  neuro =~ negative_affect + self_reproach
  # self-compassion
  nsc =~ self_judgment + isolation + over_identification
  psc =~ self_kindness + common_humanity + mindfulness
  # regressions
  ptgr ~ cope + soc + neuro
  ptss ~ cope + soc + neuro
# residula correlations
  self_judgment ~~ self_kindness
"
\end{verbatim}

\newpage

\hypertarget{model-5}{%
\section{Model 5}\label{model-5}}

M5 remove only the regression effect of the negative component of
self-compassion from M4.

\begin{verbatim}
model5 <- "
  # post-traumatic growth
  ptgr =~ life_appreciation + new_possibilities + 
         personal_strength + spirituality_changes + 
         interpersonal_relationships
  # ptsd
  ptss =~ avoiding + intrusivity + iperarousal
  # coping
  cope =~ positive_attitude + problem_orientation 
  # perceived social support
  soc =~ family + friends + significant_other
  # self-compassion
  nsc =~ self_judgment + isolation + over_identification
  psc =~ self_kindness + common_humanity + mindfulness
  # neuroticism
  neuro =~ negative_affect + self_reproach
  # regressions
  ptgr ~ cope + soc + psc + neuro
  ptss ~ cope + soc + psc + neuro
  # residual correlations
  self_judgment ~~ self_kindness
"
\end{verbatim}

\newpage

\hypertarget{model-6}{%
\section{Model 6}\label{model-6}}

M6 removes only the positive component of self-compassion from M4.

\newpage

\hypertarget{model-7}{%
\section{Model 7}\label{model-7}}

M7: mediation model with the two components of self-compassion

\begin{verbatim}
model7 <- "
  # post-traumatic growth
  ptgr =~ life_appreciation + new_possibilities + 
         personal_strength + spirituality_changes + 
         interpersonal_relationships
  # ptsd
  ptss =~ avoiding + intrusivity + iperarousal
  # coping
  cope =~ positive_attitude + problem_orientation 
  # perceived social support
  soc =~ family + friends + significant_other
  # self-compassion
  nsc =~ self_judgment + isolation + over_identification
  psc =~ self_kindness + common_humanity + mindfulness
  # neuroticism
  neuro =~ negative_affect + self_reproach
  # regressions
  ptgr ~ dg_cope*cope + dg_soc*soc + dg_neuro*neuro
  ptss ~ ds_cope*cope + ds_soc*soc + ds_neuro*neuro
  nsc ~ nsc_cope*cope + nsc_soc*soc + nsc_neuro*neuro
  psc ~ psc_cope*cope + psc_soc*soc + psc_neuro*neuro
  ptgr ~ ig_nsc*nsc + ig_psc*psc 
  ptss ~ is_nsc*nsc + is_psc*psc
  # residual correlations
  self_judgment ~~ self_kindness
  # indirect and total effects
  # cope
  i_cope_s := nsc_cope * is_nsc + psc_cope * is_psc
  i_cope_g := nsc_cope * ig_nsc + psc_cope * ig_psc
  tot_cope_s := i_cope_s + ds_cope
  tot_cope_g := i_cope_g + dg_cope
  tot_cope := i_cope_s + i_cope_g
  # soc
  i_soc_s := nsc_soc * is_nsc + psc_soc * is_psc
  i_soc_g := nsc_soc * ig_nsc + psc_soc * ig_psc
  tot_soc_s := i_soc_s + ds_soc
  tot_soc_g := i_soc_g + dg_soc
  tot_soc := i_soc_s + i_soc_g
  # neuro
  i_neuro_s := nsc_neuro * is_nsc + psc_neuro * is_psc
  i_neuro_g := nsc_neuro * ig_nsc + psc_neuro * ig_psc
  tot_neuro_s := i_neuro_s + ds_neuro
  tot_neuro_g := i_neuro_g + dg_neuro
  tot_neuro := i_neuro_s + i_neuro_g
"
\end{verbatim}

\newpage

\hypertarget{refs}{}
\begin{CSLReferences}{1}{0}
\leavevmode\hypertarget{ref-garnefski2008post}{}%
Garnefski, N., Kraaij, V., Schroevers, M., \& Somsen, G. (2008).
Post-traumatic growth after a myocardial infarction: A matter of
personality, psychological health, or cognitive coping? \emph{Journal of
Clinical Psychology in Medical Settings}, \emph{15}(4), 270--277.

\leavevmode\hypertarget{ref-inoue2006psychological}{}%
Inoue, M., Tsukano, K., Muraoka, M., Kaneko, F., \& Okamura, H. (2006).
Psychological impact of verbal abuse and violence by patients on nurses
working in psychiatric departments. \emph{Psychiatry and Clinical
Neurosciences}, \emph{60}(1), 29--36.

\leavevmode\hypertarget{ref-neff2007examination}{}%
Neff, K. D., Rude, S. S., \& Kirkpatrick, K. L. (2007). An examination
of self-compassion in relation to positive psychological functioning and
personality traits. \emph{Journal of Research in Personality},
\emph{41}(4), 908--916.

\leavevmode\hypertarget{ref-neff2019examining}{}%
Neff, K. D., Tóth-Király, I., Yarnell, L. M., Arimitsu, K., Castilho,
P., Ghorbani, N., Guo, H. X., Hirsch, J. K., Hupfeld, J., Hutz, C. S.,
\& others. (2019). Examining the factor structure of the self-compassion
scale in 20 diverse samples: Support for use of a total score and six
subscale scores. \emph{Psychological Assessment}, \emph{31}(1), 27--45.

\leavevmode\hypertarget{ref-ullrich2020use}{}%
Ullrich-French, S., \& Cox, A. E. (2020). The use of latent profiles to
explore the multi-dimensionality of self-compassion. \emph{Mindfulness},
\emph{11}, 1483--1499.

\leavevmode\hypertarget{ref-wu2005posttraumatic}{}%
Wu, K. K., Chan, S. K., \& Ma, T. M. (2005). Posttraumatic stress,
anxiety, and depression in survivors of severe acute respiratory
syndrome (SARS). \emph{Journal of Traumatic Stress: Official Publication
of The International Society for Traumatic Stress Studies},
\emph{18}(1), 39--42.

\leavevmode\hypertarget{ref-yarnell2019gender}{}%
Yarnell, L. M., Neff, K. D., Davidson, O. A., \& Mullarkey, M. (2019).
Gender differences in self-compassion: Examining the role of gender role
orientation. \emph{Mindfulness}, \emph{10}(6), 1136--1152.

\end{CSLReferences}
%\end{appendix}
